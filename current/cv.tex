% Copyright 2012 Sean Leather (sean.leather@gmail.com)

\documentclass[11pt,a4paper,roman]{moderncv}

%-------------------------------------------------------------------------------

% Style
\moderncvstyle{casual}
\moderncvcolor{blue}

% Adjust the page margins
\usepackage[scale=0.75]{geometry}
\setlength{\hintscolumnwidth}{22ex} % width of first column

% For \xspace: checks for space after command
\usepackage{xspace}

% For ordinal numbers, use \nth{1} for 1st
\usepackage[super]{nth}

%-------------------------------------------------------------------------------

% Personal information
\firstname{Sean}
\familyname{Leather}
\email{sean.leather@gmail.com}

% TODO: Import file with other personal information.

%-------------------------------------------------------------------------------

\begin{document}

% Make title
\makecvtitle

%-------------------------------------------------------------------------------

% Commands for abbreviations

% See above
\newcommand\seeabovedesc{(\textit{See description above.})}

% Cities
\newcommand\utrecht{Utrecht, The Netherlands}
\newcommand\austin{Austin, Texas, USA}
\newcommand\stlouis{St. Louis, Missouri, USA}
\newcommand\chippewafalls{Chippewa Falls, Wisconsin, USA}
\newcommand\whitemarsh{White Marsh, Maryland, USA}
\newcommand\altamont{Altamont, Tennessee, USA}
\newcommand\chattanooga{Chattanooga, Tennessee, USA}
\newcommand\london{London, UK}
\newcommand\centurion{Centurion, South Africa}

% Universities
\newcommand\uu{Utrecht University}
\newcommand\wustl{Washington University in St. Louis}
\newcommand\ut{University of Texas at Austin}

%-------------------------------------------------------------------------------

% Programming languages

\newcommand\Cpp{C{}\texttt{++}\xspace}

%-------------------------------------------------------------------------------

\section{Education}

\cventry%
{Jan 2008 -- present}%
{PhD}%
{\uu}%
{\utrecht}%
{}%
{My research focuses on datatype-generic programming and program transformation
in functional programming languages.}

\cventry%
{Aug 2003 -- May 2006}%
{MS in Electrical and Computer Engineering}%
{University of Texas at Austin}%
{\austin}%
{\textit{3.57/4.00}}%
{My courses focused on computer architecture, VLSI, hardware simulation,
performance evaluation, compilers, and programming language theory.}

\cventry%
{Aug 1998 -- May 2003}%
{BS in Computer Science, BS in Computer Engineering}%
{\wustl}%
{\stlouis}%
{\textit{3.84/4.00}}%
{}

%-------------------------------------------------------------------------------

\section{Industrial Experience}

%-------------------------------------------------------------------------------

\subsection{Full-time}

\cventry%
{Jan 2006 -- Oct 2007}%
{Design Engineer}%
{Anue Systems}%
{\austin}%
{}%
{I developed embedded software in C/\Cpp, embedded web applications in C/HTML/JavaScript, and RTL test benches in Verilog.
\begin{itemize}
\item Designed low-level libraries with clean APIs for AVR microcontrollers, custom FPGAs, real-time clocks, etc.
\item Designed an extensive Verilog test bench for the custom RTL of a 10-gigabit Ethernet emulator
\item Introduced AJAX and the use of JavaScript libraries into the web interface development for embedded systems
\end{itemize}}

%-------------------------------------------------------------------------------

\subsection{Internships and Co-ops}

\cventry%
{May -- Aug 2005}%
{Performance Modeling Engineer}%
{IBM}%
{\austin}%
{}%
{I worked on the next-generation PowerPC architecture performance model.
\begin{itemize}
\item Developed a tool to improve the modeling methodology by screening the most important parameters with minimal simulations
\item Evaluated simultaneous multithreading (SMT) performance to find a speedup of several points
\end{itemize}}

\cventry%
{May -- Aug 2004}%
{Architecture Design Engineer}%
{Cray}%
{\chippewafalls}%
{}%
{I worked on the vector instruction scheduler for a mixed vector/scalar architecture.
\begin{itemize}
\item Developed tools for analyzing vector register dataflow in benchmark kernels
\item Modeled approaches to scheduling vector instructions to multiple on-chip clusters
\end{itemize}}

\cventry%
{Jun -- Aug 2003}%
{Digital Hardware Design Engineer}%
{Rockwell Collins}%
{\whitemarsh}%
{}%
{I implemented interfacing logic in Verilog for controlling high-performance frequency conversion RTL in FPGAs.}

\cventry%
{May -- Aug 2002}%
{Avionics Research and Development Engineer}%
{Boeing}%
{\stlouis}%
{}%
{I implemented a driver for touchscreen monitors in an embedded, real-time operating system (RTOS).}

\cventry%
{Aug -- Dec 2001}%
{Avionics Test Engineer}%
{Boeing}%
{\stlouis}%
{}%
{I tested, debugged, and repaired electronics and firmware on single-board computers for the AV-8B avionics test bench.}

\cventry%
{Jan -- May 2001}%
{Embedded Software Engineer}%
{Boeing}%
{\stlouis}%
{}%
{I designed, implemented, tested, and documented a cross-platform \Cpp framework for logging flight simulation network data on Irix, Windows, Solaris, and a Green Hills RTOS.}

\cventry%
{May -- Aug 1999}%
{Web Applications Developer}%
{Higher Technology Services}%
{\chattanooga}%
{}%
{I managed PC and SPARC Linux servers and created marketing-oriented data-driven websites for customers in Perl.}

%-------------------------------------------------------------------------------

\subsection{Miscellaneous}

\cventry%
{May -- Jul 2001}%
{Waterfront Director}%
{Skymont Scout Reservation}%
{\altamont}%
{}%
{I managed activities and staff in the aquatics area at the Boy Scout summer camp.
\begin{itemize}
\item Trained instructors for rowing, canoeing, and swimming
\item Instructed Scouts in lifeguarding
\end{itemize}}

%-------------------------------------------------------------------------------

\section{Academic Experience}

\newcommand\coursenote{\scriptsize [U]ndergraduate and [G]raduate courses (with year)}

%-------------------------------------------------------------------------------

\subsection{Teaching}

\cvlistitem{\coursenote}

\cventry%
{}%
{Languages and Compilers (INFOB3TC)}%
{\uu}%
{}%
{U(2)}%
{Formal languages, grammars, parsing, and code generation with Haskell.
\begin{itemize}
\item Nov -- Dec 2012
\item Nov 2011 -- Jan 2012
\end{itemize}}

\cventry%
{}%
{Generic Programming (INFOGP)}%
{\uu}%
{}%
{G}%
{Datatype-generic programming with Haskell and Agda.
\begin{itemize}
\item Sep -- Nov 2012
\item Sep -- Nov 2011
\end{itemize}}

%-------------------------------------------------------------------------------

\subsection{Teaching Assistant}

\cvlistitem{\coursenote}

\cventry%
{}%
{Generic Programming (INFOGP)}%
{\uu}%
{}%
{}%
{\begin{itemize}
\item Sep - Nov 2010
\item Apr - Jul 2009
\end{itemize}}

\cventry%
{}%
{Functional Programming (INFOFP)}%
{\uu}%
{}%
{U(2)}%
{Basics, lists, recursion, I/O, data structures, type classes, reasoning with Haskell.
\begin{itemize}
\item Nov 2009 -- Feb 2010
\end{itemize}}

\cventry%
{}%
{Languages and Compilers (INFOB3TC)}%
{\uu}%
{}%
{}%
{\begin{itemize}
\item Nov 2008 -- Jan 2009
\end{itemize}}

\cventry%
{}%
{Software Engineering (INFOSWE)}%
{\uu}%
{}%
{G}%
{Software life cycle, formal requirements, specification with Z, build/version management, aspect-oriented programming, and testing.
\begin{itemize}
\item Sep - Nov 2008
\end{itemize}}

\cventry%
{}%
{Computer Science II (CS~102G)}%
{\wustl}%
{}%
{U(1)}%
{Object-oriented programming concepts, exceptions, GUIs, threads, concurrency, synchronization in Java.
\begin{itemize}
\item Jan -- Apr 2000
\end{itemize}}

\cventry%
{}%
{Programming Concepts and Practice (CS~504N)}%
{\wustl}%
{}%
{G(1)}%
{Object-oriented programming concepts, threads, basic algorithms in Java.
\begin{itemize}
\item Aug -- Dec 1999
\end{itemize}}

%-------------------------------------------------------------------------------

\section{Advising, \uu}

\subsection{Master's Thesis, Completed}

\cvitem{Aug 2009}{Eelco Lempsik (co-advised with Andres Löh): Generic Type-Safe \texttt{diff} and \texttt{patch} for Families of Datatypes}

\subsection{Master's Thesis, Current}

\cvitem{}{Joeri van Eekelen (co-advised with Johan Jeuring)}
\cvitem{}{Tom Hastjarjanto (co-advised with Johan Jeuring)}
\cvitem{}{Bram Schuur (co-advised with Johan Jeuring)}

\subsection{Experimentation Projects}

\cvitem{2012}{Thijs Alkemade: Holes in GHC}
\cvitem{2011}{Jurriën Stutterheim: Datatype-Generic Generation}

%-------------------------------------------------------------------------------

% Publications

\nocite{*}
\bibliographystyle{unsrt}
\bibliography{publications}

%-------------------------------------------------------------------------------

\section{Talks}

\cventry%
{Jun 2012}%
{Generic Programming in Haskell}%
{Lambda Luminaries}%
{\centurion}%
{}%
{Two-part talk introducing datatype-generic programming and describing ``xformat''}

\cventry%
{Nov 2011}%
{Generic Deriving in GHC 7.2}%
{Dutch Haskell Users Group}%
{\utrecht}%
{}%
{Motivation behind and design of the ``generic-deriving'' library included with GHC 7.2}

\cventry%
{Nov 2009}%
{Extensibility and Type Safety in Formatting: The Design of ``xformat''}%
{Dutch Haskell Users Group}%
{\utrecht}%
{}%
{Motivation behind and design of a library aimed at solving the ``\texttt{printf} problem''}

\cventry%
{Jul 2009}%
{Fun and Generic Things to Do with EMGM}%
{London Haskell Users Group}%
{\london}%
{}%
{Writing and using generic functions with the library ``Extensible and Modular Generics for the Masses''}

%-------------------------------------------------------------------------------

\section{Professional Organizations}

\cvitem{1999 -- present}{Member of ACM
\begin{description}
\item 1999, 2000: Treasurer of student chapter, \wustl
\item 2001: President of student chapter, \wustl
\end{description}}

\cvitem{2002 -- 2011}{Member of IEEE
\begin{description}
\item 2002 -- 2003: Vice president of student chapter, \wustl
\end{description}}

%-------------------------------------------------------------------------------

\section{Awards}

\cvitem{2003}{Engineering Doctoral Fellowship, \ut}
\cvitem{2003}{MCD Fellowship, \ut}
\cvitem{1998}{Eagle Scout}

\end{document}

